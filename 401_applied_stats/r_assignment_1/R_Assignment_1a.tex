% Options for packages loaded elsewhere
\PassOptionsToPackage{unicode}{hyperref}
\PassOptionsToPackage{hyphens}{url}
%
\documentclass[
]{article}
\usepackage{amsmath,amssymb}
\usepackage{lmodern}
\usepackage{iftex}
\ifPDFTeX
  \usepackage[T1]{fontenc}
  \usepackage[utf8]{inputenc}
  \usepackage{textcomp} % provide euro and other symbols
\else % if luatex or xetex
  \usepackage{unicode-math}
  \defaultfontfeatures{Scale=MatchLowercase}
  \defaultfontfeatures[\rmfamily]{Ligatures=TeX,Scale=1}
\fi
% Use upquote if available, for straight quotes in verbatim environments
\IfFileExists{upquote.sty}{\usepackage{upquote}}{}
\IfFileExists{microtype.sty}{% use microtype if available
  \usepackage[]{microtype}
  \UseMicrotypeSet[protrusion]{basicmath} % disable protrusion for tt fonts
}{}
\makeatletter
\@ifundefined{KOMAClassName}{% if non-KOMA class
  \IfFileExists{parskip.sty}{%
    \usepackage{parskip}
  }{% else
    \setlength{\parindent}{0pt}
    \setlength{\parskip}{6pt plus 2pt minus 1pt}}
}{% if KOMA class
  \KOMAoptions{parskip=half}}
\makeatother
\usepackage{xcolor}
\usepackage[margin=1in]{geometry}
\usepackage{color}
\usepackage{fancyvrb}
\newcommand{\VerbBar}{|}
\newcommand{\VERB}{\Verb[commandchars=\\\{\}]}
\DefineVerbatimEnvironment{Highlighting}{Verbatim}{commandchars=\\\{\}}
% Add ',fontsize=\small' for more characters per line
\usepackage{framed}
\definecolor{shadecolor}{RGB}{248,248,248}
\newenvironment{Shaded}{\begin{snugshade}}{\end{snugshade}}
\newcommand{\AlertTok}[1]{\textcolor[rgb]{0.94,0.16,0.16}{#1}}
\newcommand{\AnnotationTok}[1]{\textcolor[rgb]{0.56,0.35,0.01}{\textbf{\textit{#1}}}}
\newcommand{\AttributeTok}[1]{\textcolor[rgb]{0.77,0.63,0.00}{#1}}
\newcommand{\BaseNTok}[1]{\textcolor[rgb]{0.00,0.00,0.81}{#1}}
\newcommand{\BuiltInTok}[1]{#1}
\newcommand{\CharTok}[1]{\textcolor[rgb]{0.31,0.60,0.02}{#1}}
\newcommand{\CommentTok}[1]{\textcolor[rgb]{0.56,0.35,0.01}{\textit{#1}}}
\newcommand{\CommentVarTok}[1]{\textcolor[rgb]{0.56,0.35,0.01}{\textbf{\textit{#1}}}}
\newcommand{\ConstantTok}[1]{\textcolor[rgb]{0.00,0.00,0.00}{#1}}
\newcommand{\ControlFlowTok}[1]{\textcolor[rgb]{0.13,0.29,0.53}{\textbf{#1}}}
\newcommand{\DataTypeTok}[1]{\textcolor[rgb]{0.13,0.29,0.53}{#1}}
\newcommand{\DecValTok}[1]{\textcolor[rgb]{0.00,0.00,0.81}{#1}}
\newcommand{\DocumentationTok}[1]{\textcolor[rgb]{0.56,0.35,0.01}{\textbf{\textit{#1}}}}
\newcommand{\ErrorTok}[1]{\textcolor[rgb]{0.64,0.00,0.00}{\textbf{#1}}}
\newcommand{\ExtensionTok}[1]{#1}
\newcommand{\FloatTok}[1]{\textcolor[rgb]{0.00,0.00,0.81}{#1}}
\newcommand{\FunctionTok}[1]{\textcolor[rgb]{0.00,0.00,0.00}{#1}}
\newcommand{\ImportTok}[1]{#1}
\newcommand{\InformationTok}[1]{\textcolor[rgb]{0.56,0.35,0.01}{\textbf{\textit{#1}}}}
\newcommand{\KeywordTok}[1]{\textcolor[rgb]{0.13,0.29,0.53}{\textbf{#1}}}
\newcommand{\NormalTok}[1]{#1}
\newcommand{\OperatorTok}[1]{\textcolor[rgb]{0.81,0.36,0.00}{\textbf{#1}}}
\newcommand{\OtherTok}[1]{\textcolor[rgb]{0.56,0.35,0.01}{#1}}
\newcommand{\PreprocessorTok}[1]{\textcolor[rgb]{0.56,0.35,0.01}{\textit{#1}}}
\newcommand{\RegionMarkerTok}[1]{#1}
\newcommand{\SpecialCharTok}[1]{\textcolor[rgb]{0.00,0.00,0.00}{#1}}
\newcommand{\SpecialStringTok}[1]{\textcolor[rgb]{0.31,0.60,0.02}{#1}}
\newcommand{\StringTok}[1]{\textcolor[rgb]{0.31,0.60,0.02}{#1}}
\newcommand{\VariableTok}[1]{\textcolor[rgb]{0.00,0.00,0.00}{#1}}
\newcommand{\VerbatimStringTok}[1]{\textcolor[rgb]{0.31,0.60,0.02}{#1}}
\newcommand{\WarningTok}[1]{\textcolor[rgb]{0.56,0.35,0.01}{\textbf{\textit{#1}}}}
\usepackage{graphicx}
\makeatletter
\def\maxwidth{\ifdim\Gin@nat@width>\linewidth\linewidth\else\Gin@nat@width\fi}
\def\maxheight{\ifdim\Gin@nat@height>\textheight\textheight\else\Gin@nat@height\fi}
\makeatother
% Scale images if necessary, so that they will not overflow the page
% margins by default, and it is still possible to overwrite the defaults
% using explicit options in \includegraphics[width, height, ...]{}
\setkeys{Gin}{width=\maxwidth,height=\maxheight,keepaspectratio}
% Set default figure placement to htbp
\makeatletter
\def\fps@figure{htbp}
\makeatother
\setlength{\emergencystretch}{3em} % prevent overfull lines
\providecommand{\tightlist}{%
  \setlength{\itemsep}{0pt}\setlength{\parskip}{0pt}}
\setcounter{secnumdepth}{-\maxdimen} % remove section numbering
\ifLuaTeX
  \usepackage{selnolig}  % disable illegal ligatures
\fi
\IfFileExists{bookmark.sty}{\usepackage{bookmark}}{\usepackage{hyperref}}
\IfFileExists{xurl.sty}{\usepackage{xurl}}{} % add URL line breaks if available
\urlstyle{same} % disable monospaced font for URLs
\hypersetup{
  pdftitle={R Assignment \#1a (50 points)},
  hidelinks,
  pdfcreator={LaTeX via pandoc}}

\title{R Assignment \#1a (50 points)}
\author{}
\date{\vspace{-2.5em}}

\begin{document}
\maketitle

\hypertarget{instructions}{%
\subsubsection{Instructions}\label{instructions}}

R markdown is a plain-text file format for integrating text and R code,
and creating transparent, reproducible and interactive reports. An R
markdown file (.Rmd) contains metadata, markdown and R code ``chunks'',
and can be ``knit'' into numerous output types. Answer the test
questions by adding R code to the fenced code areas below each item.
Once completed, you will ``knit'' and submit the resulting .html file,
as well the .Rmd file. The .html will include your R code \emph{and} the
output.

\textbf{Before proceeding, look to the top of the .Rmd for the (YAML)
metadata block, where the \emph{title} and \emph{output} are given.
Please change \emph{title} from `Programming with R Assignment \#1' to
your name, with the format `lastName\_firstName.'}

If you encounter issues with knitting the .html, please send an email
via Canvas to your TA.

Each code chunk is delineated by six (6) backticks; three (3) at the
start and three (3) at the end. After the opening ticks, arguments are
passed to the code chunk and in curly brackets. \textbf{Please do not
add or remove backticks, or modify the arguments or values inside the
curly brackets}. An example code chunk is included here:

\begin{Shaded}
\begin{Highlighting}[]
\CommentTok{\# Comments are included in each code chunk, simply as prompts}

\NormalTok{...R code placed here}

\NormalTok{...R code placed here}
\end{Highlighting}
\end{Shaded}

You need only enter text inside the code chunks for each test item.

Depending on the problem, grading will be based on: 1) the correct
result, 2) coding efficiency and 3) graphical presentation features
(labeling, colors, size, legibility, etc). I will be looking for
well-rendered displays. In the ``knit'' document, only those results
specified in the problem statements should be displayed. For example, do
not output - i.e.~send to the Console - the contents of vectors or data
frames unless requested by the problem. You should be able to code for
each solution in fewer than ten lines; though code for your
visualizations may exceed this.

\textbf{Submit both the .Rmd and .html files for grading}

\begin{center}\rule{0.5\linewidth}{0.5pt}\end{center}

\textbf{Example Problem with Solution:} Use \emph{rbinom()} to generate
two random samples of size 10,000 from the binomial distribution. For
the first sample, use p = 0.45 and n = 10. For the second sample, use p
= 0.55 and n = 10.

\begin{enumerate}
\def\labelenumi{(\alph{enumi})}
\tightlist
\item
  Convert the sample frequencies to sample proportions and compute the
  mean number of successes for each sample. Present these statistics.
\end{enumerate}

\begin{Shaded}
\begin{Highlighting}[]
\FunctionTok{set.seed}\NormalTok{(}\DecValTok{123}\NormalTok{)}
\NormalTok{sample.one }\OtherTok{\textless{}{-}} \FunctionTok{table}\NormalTok{(}\FunctionTok{rbinom}\NormalTok{(}\DecValTok{10000}\NormalTok{, }\DecValTok{10}\NormalTok{, }\FloatTok{0.45}\NormalTok{)) }\SpecialCharTok{/} \DecValTok{10000}
\NormalTok{sample.two }\OtherTok{\textless{}{-}} \FunctionTok{table}\NormalTok{(}\FunctionTok{rbinom}\NormalTok{(}\DecValTok{10000}\NormalTok{, }\DecValTok{10}\NormalTok{, }\FloatTok{0.55}\NormalTok{)) }\SpecialCharTok{/} \DecValTok{10000}

\NormalTok{successes }\OtherTok{\textless{}{-}} \FunctionTok{seq}\NormalTok{(}\DecValTok{0}\NormalTok{, }\DecValTok{10}\NormalTok{)}

\FunctionTok{sum}\NormalTok{(sample.one }\SpecialCharTok{*}\NormalTok{ successes) }\CommentTok{\# [1] 4.4827}
\end{Highlighting}
\end{Shaded}

\begin{verbatim}
## [1] 4.4827
\end{verbatim}

\begin{Shaded}
\begin{Highlighting}[]
\FunctionTok{sum}\NormalTok{(sample.two }\SpecialCharTok{*}\NormalTok{ successes) }\CommentTok{\# [1] 5.523}
\end{Highlighting}
\end{Shaded}

\begin{verbatim}
## [1] 5.523
\end{verbatim}

\begin{enumerate}
\def\labelenumi{(\alph{enumi})}
\setcounter{enumi}{1}
\tightlist
\item
  Present the proportions in a vertical, side-by-side barplot color
  coding the two samples.
\end{enumerate}

\begin{Shaded}
\begin{Highlighting}[]
\NormalTok{counts }\OtherTok{\textless{}{-}} \FunctionTok{rbind}\NormalTok{(sample.one, sample.two)}

\FunctionTok{barplot}\NormalTok{(counts, }\AttributeTok{main =} \StringTok{"Comparison of Binomial Sample Proportions"}\NormalTok{, }
  \AttributeTok{ylab =} \StringTok{"Frequency"}\NormalTok{, }\AttributeTok{ylim =} \FunctionTok{c}\NormalTok{(}\DecValTok{0}\NormalTok{,}\FloatTok{0.3}\NormalTok{),}\AttributeTok{xlab =} \StringTok{"Number of Successes"}\NormalTok{,}
  \AttributeTok{beside =} \ConstantTok{TRUE}\NormalTok{, }\AttributeTok{col =} \FunctionTok{c}\NormalTok{(}\StringTok{"darkblue"}\NormalTok{,}\StringTok{"red"}\NormalTok{),}\AttributeTok{legend =} \FunctionTok{rownames}\NormalTok{(counts),}
  \AttributeTok{names.arg =} \FunctionTok{c}\NormalTok{(}\StringTok{"0"}\NormalTok{, }\StringTok{"1"}\NormalTok{, }\StringTok{"2"}\NormalTok{, }\StringTok{"3"}\NormalTok{, }\StringTok{"4"}\NormalTok{, }\StringTok{"5"}\NormalTok{, }\StringTok{"6"}\NormalTok{, }\StringTok{"7"}\NormalTok{, }\StringTok{"8"}\NormalTok{, }\StringTok{"9"}\NormalTok{, }\StringTok{"10"}\NormalTok{))}
\end{Highlighting}
\end{Shaded}

\includegraphics{R_Assignment_1a_files/figure-latex/testExampleB-1.pdf}

\hypertarget{please-delete-the-instructions-and-examples-shown-above-prior-to-submitting-your-.rmd-and-.html-files.}{%
\paragraph{Please delete the Instructions and Examples shown above prior
to submitting your .Rmd and .html
files.}\label{please-delete-the-instructions-and-examples-shown-above-prior-to-submitting-your-.rmd-and-.html-files.}}

\begin{center}\rule{0.5\linewidth}{0.5pt}\end{center}

\hypertarget{test-items-starts-from-here---there-are-5-sections---50-points-total}{%
\subsubsection{Test Items starts from here - There are 5 sections - 50
points
total}\label{test-items-starts-from-here---there-are-5-sections---50-points-total}}

Read each question carefully and address each element. Do not output
contents of vectors or data frames unless requested.

\hypertarget{section-1-8-points-this-problem-deals-with-vector-manipulations.}{%
\subparagraph{Section 1: (8 points) This problem deals with vector
manipulations.}\label{section-1-8-points-this-problem-deals-with-vector-manipulations.}}

(1)(a) Create a vector that contains the following, in this order, and
output the final, resulting vector. Do not round any values, unless
requested. * A sequence of integers from 0 to 4, inclusive. * The number
13 * Three repetitions of the vector c(2, -5.1, -23). * The arithmetic
sum of 7/42, 3 and 35/42

\begin{Shaded}
\begin{Highlighting}[]
\NormalTok{triple\_peat }\OtherTok{\textless{}{-}} \FunctionTok{c}\NormalTok{(}\DecValTok{2}\NormalTok{,}\SpecialCharTok{{-}}\FloatTok{5.1}\NormalTok{,}\SpecialCharTok{{-}}\DecValTok{23}\NormalTok{)}

\NormalTok{arith\_vector }\OtherTok{\textless{}{-}} \FunctionTok{sum}\NormalTok{(}\FunctionTok{c}\NormalTok{((}\DecValTok{7}\SpecialCharTok{/}\DecValTok{42}\NormalTok{),}\DecValTok{3}\NormalTok{,(}\DecValTok{35}\SpecialCharTok{/}\DecValTok{42}\NormalTok{)))}

\NormalTok{test1a\_vector }\OtherTok{\textless{}{-}} \FunctionTok{c}\NormalTok{(}\DecValTok{0}\SpecialCharTok{:}\DecValTok{4}\NormalTok{,}\DecValTok{13}\NormalTok{,}\FunctionTok{rep}\NormalTok{(triple\_peat, }\AttributeTok{times =} \DecValTok{3}\NormalTok{),arith\_vector)}
\NormalTok{test1a\_vector}
\end{Highlighting}
\end{Shaded}

\begin{verbatim}
##  [1]   0.0   1.0   2.0   3.0   4.0  13.0   2.0  -5.1 -23.0   2.0  -5.1 -23.0
## [13]   2.0  -5.1 -23.0   4.0
\end{verbatim}

(1)(b) Sort the vector created in (1)(a) in ascending order. Output this
result. Determine the length of the resulting vector and assign to
``L''. Output L. Generate a descending sequence starting with L and
ending with 1. Add this descending sequence arithmetically the sorted
vector. This is vector addition, not vector combination. Output the
contents. Do not round any values.

\begin{Shaded}
\begin{Highlighting}[]
\NormalTok{sorted\_vector }\OtherTok{\textless{}{-}} \FunctionTok{sort}\NormalTok{(test1a\_vector,}\AttributeTok{decreasing =} \ConstantTok{FALSE}\NormalTok{)}
\NormalTok{sorted\_vector}
\end{Highlighting}
\end{Shaded}

\begin{verbatim}
##  [1] -23.0 -23.0 -23.0  -5.1  -5.1  -5.1   0.0   1.0   2.0   2.0   2.0   2.0
## [13]   3.0   4.0   4.0  13.0
\end{verbatim}

\begin{Shaded}
\begin{Highlighting}[]
\NormalTok{L }\OtherTok{\textless{}{-}} \FunctionTok{length}\NormalTok{(sorted\_vector)}
\NormalTok{L}
\end{Highlighting}
\end{Shaded}

\begin{verbatim}
## [1] 16
\end{verbatim}

\begin{Shaded}
\begin{Highlighting}[]
\NormalTok{desc\_seq }\OtherTok{\textless{}{-}}\NormalTok{ L}\SpecialCharTok{:}\DecValTok{1}
\NormalTok{desc\_seq}
\end{Highlighting}
\end{Shaded}

\begin{verbatim}
##  [1] 16 15 14 13 12 11 10  9  8  7  6  5  4  3  2  1
\end{verbatim}

\begin{Shaded}
\begin{Highlighting}[]
\NormalTok{added\_vector }\OtherTok{\textless{}{-}}\NormalTok{ sorted\_vector }\SpecialCharTok{+}\NormalTok{ desc\_seq}
\NormalTok{added\_vector}
\end{Highlighting}
\end{Shaded}

\begin{verbatim}
##  [1] -7.0 -8.0 -9.0  7.9  6.9  5.9 10.0 10.0 10.0  9.0  8.0  7.0  7.0  7.0  6.0
## [16] 14.0
\end{verbatim}

(1)(c) Extract the first and last elements of the vector you have
created in (1)(b) to form another vector of the extracted elements. Form
a third vector from the elements not extracted. Output these vectors.

\begin{Shaded}
\begin{Highlighting}[]
\NormalTok{extract\_vector }\OtherTok{\textless{}{-}} \FunctionTok{c}\NormalTok{(added\_vector[}\DecValTok{1}\NormalTok{],}\FunctionTok{tail}\NormalTok{(added\_vector,}\AttributeTok{n=}\DecValTok{1}\NormalTok{))}
\NormalTok{extract\_vector}
\end{Highlighting}
\end{Shaded}

\begin{verbatim}
## [1] -7 14
\end{verbatim}

\begin{Shaded}
\begin{Highlighting}[]
\NormalTok{unextract\_vector }\OtherTok{\textless{}{-}} \FunctionTok{c}\NormalTok{(added\_vector[}\DecValTok{2}\SpecialCharTok{:}\NormalTok{(}\FunctionTok{length}\NormalTok{(added\_vector)}\SpecialCharTok{{-}}\DecValTok{1}\NormalTok{)])}
\NormalTok{unextract\_vector}
\end{Highlighting}
\end{Shaded}

\begin{verbatim}
##  [1] -8.0 -9.0  7.9  6.9  5.9 10.0 10.0 10.0  9.0  8.0  7.0  7.0  7.0  6.0
\end{verbatim}

(1)(d) Use the vectors from (c) to reconstruct the vector in (b). Output
this vector. Sum the elements.

\begin{Shaded}
\begin{Highlighting}[]
\NormalTok{reconstructed\_vector }\OtherTok{\textless{}{-}} \FunctionTok{c}\NormalTok{(extract\_vector[}\DecValTok{1}\NormalTok{],unextract\_vector,extract\_vector[}\SpecialCharTok{{-}}\DecValTok{1}\NormalTok{])}
\NormalTok{reconstructed\_vector}
\end{Highlighting}
\end{Shaded}

\begin{verbatim}
##  [1] -7.0 -8.0 -9.0  7.9  6.9  5.9 10.0 10.0 10.0  9.0  8.0  7.0  7.0  7.0  6.0
## [16] 14.0
\end{verbatim}

\begin{Shaded}
\begin{Highlighting}[]
\NormalTok{summed\_reconstruct }\OtherTok{\textless{}{-}} \FunctionTok{sum}\NormalTok{(reconstructed\_vector)}
\NormalTok{summed\_reconstruct}
\end{Highlighting}
\end{Shaded}

\begin{verbatim}
## [1] 84.7
\end{verbatim}

\begin{center}\rule{0.5\linewidth}{0.5pt}\end{center}

\hypertarget{section-2-10-points-the-expression-y-sinx2-cosx2-is-a-trigonometric-function.}{%
\subparagraph{Section 2: (10 points) The expression y = sin(x/2) +
cos(x/2) is a trigonometric
function.}\label{section-2-10-points-the-expression-y-sinx2-cosx2-is-a-trigonometric-function.}}

(2)(a) Create a user-defined function - via \emph{function()} - that
implements the trigonometric function above, accepts numeric values,
``x,'' calculates and returns values ``y.''

\begin{Shaded}
\begin{Highlighting}[]
\NormalTok{test2a\_trig\_func }\OtherTok{\textless{}{-}} \ControlFlowTok{function}\NormalTok{(x) \{}
\NormalTok{  y }\OtherTok{\textless{}{-}} \FunctionTok{sin}\NormalTok{(x}\SpecialCharTok{/}\DecValTok{2}\NormalTok{) }\SpecialCharTok{+} \FunctionTok{cos}\NormalTok{(x}\SpecialCharTok{/}\DecValTok{2}\NormalTok{)}
  \FunctionTok{return}\NormalTok{(y)}
\NormalTok{\}}

\CommentTok{\# for testing, the following should return}
\CommentTok{\# 0.4931506}
\CommentTok{\# test2a\_trig\_func(4)}
\end{Highlighting}
\end{Shaded}

(2)(b) Create a vector, x, of 4001 equally-spaced values from -2 to 2,
inclusive. Compute values for y using the vector x and your function
from (2)(a). \textbf{Do not output x or y.} Find the value in the vector
x that corresponds to the maximum value in the vector y. Restrict
attention to only the values of x and y you have computed; i.e.~do not
interpolate. Round to 3 decimal places and output both the maximum y and
corresponding x value.

Finding the two desired values can be accomplished in as few as two
lines of code. Do not use packages or programs you may find on the
internet or elsewhere. Do not output the other elements of the vectors x
and y. Relevant coding methods are given in the \emph{Quick Start Guide
for R}.

\begin{Shaded}
\begin{Highlighting}[]
\FunctionTok{library}\NormalTok{(glue)}

\NormalTok{x }\OtherTok{\textless{}{-}} \FunctionTok{c}\NormalTok{(}\FunctionTok{seq}\NormalTok{(}\SpecialCharTok{{-}}\DecValTok{2}\NormalTok{, }\DecValTok{2}\NormalTok{, }\AttributeTok{length.out =} \DecValTok{4001}\NormalTok{))}
\NormalTok{y }\OtherTok{\textless{}{-}} \FunctionTok{unlist}\NormalTok{(}\FunctionTok{lapply}\NormalTok{(x,}\AttributeTok{FUN=}\NormalTok{test2a\_trig\_func))}

\NormalTok{max\_y }\OtherTok{\textless{}{-}} \FunctionTok{max}\NormalTok{(y)}

\NormalTok{max\_y\_index }\OtherTok{\textless{}{-}} \FunctionTok{which}\NormalTok{(y }\SpecialCharTok{==}\NormalTok{ max\_y)}

\NormalTok{max\_x }\OtherTok{\textless{}{-}}\NormalTok{ x[max\_y\_index]}

\FunctionTok{print}\NormalTok{(}\FunctionTok{paste}\NormalTok{(}\StringTok{"Max xy = ("}\NormalTok{,max\_x,}\StringTok{\textquotesingle{},\textquotesingle{}}\NormalTok{,}\FunctionTok{round}\NormalTok{(max\_y,}\DecValTok{3}\NormalTok{),}\StringTok{\textquotesingle{})\textquotesingle{}}\NormalTok{))}
\end{Highlighting}
\end{Shaded}

\begin{verbatim}
## [1] "Max xy = ( 1.571 , 1.414 )"
\end{verbatim}

(2)(c) Plot y versus x in color, with x on the horizontal axis. Show the
location of the maximum value of y determined in 2(b). Show the values
of x and y corresponding to the maximum value of y in the display. Add a
title and other features such as text annotations. Text annotations may
be added via \emph{text()} for base R plots and \emph{geom\_text()} or
\emph{geom\_label()} for ggplots.

\begin{Shaded}
\begin{Highlighting}[]
\FunctionTok{library}\NormalTok{(tidyverse)}
\end{Highlighting}
\end{Shaded}

\begin{verbatim}
## -- Attaching packages --------------------------------------- tidyverse 1.3.2 --
## v ggplot2 3.4.0      v purrr   1.0.0 
## v tibble  3.1.8      v dplyr   1.0.10
## v tidyr   1.2.1      v stringr 1.5.0 
## v readr   2.1.3      v forcats 0.5.2 
## -- Conflicts ------------------------------------------ tidyverse_conflicts() --
## x dplyr::filter() masks stats::filter()
## x dplyr::lag()    masks stats::lag()
\end{verbatim}

\begin{Shaded}
\begin{Highlighting}[]
\NormalTok{trig\_df }\OtherTok{\textless{}{-}} \FunctionTok{data.frame}\NormalTok{(x,y)}
\NormalTok{max\_df }\OtherTok{\textless{}{-}} \FunctionTok{data.frame}\NormalTok{(max\_x,max\_y)}
\NormalTok{max\_point }\OtherTok{\textless{}{-}}\NormalTok{ trig\_df[}\FunctionTok{which.max}\NormalTok{(trig\_df}\SpecialCharTok{$}\NormalTok{y),]}


\FunctionTok{ggplot}\NormalTok{(}\AttributeTok{data=}\NormalTok{trig\_df, }\FunctionTok{aes}\NormalTok{(x, y)) }\SpecialCharTok{+} 
  \FunctionTok{geom\_point}\NormalTok{(}\AttributeTok{color =} \StringTok{\textquotesingle{}black\textquotesingle{}}\NormalTok{,}\AttributeTok{size =} \DecValTok{1}\NormalTok{) }\SpecialCharTok{+} 
  \FunctionTok{geom\_point}\NormalTok{(}\AttributeTok{data =}\NormalTok{ trig\_df[}\FunctionTok{which.max}\NormalTok{(trig\_df}\SpecialCharTok{$}\NormalTok{y),],}\AttributeTok{color=}\StringTok{\textquotesingle{}red\textquotesingle{}}\NormalTok{,}\AttributeTok{size=}\DecValTok{3}\NormalTok{) }\SpecialCharTok{+}
  \FunctionTok{geom\_text}\NormalTok{(}\AttributeTok{data =}\NormalTok{ trig\_df[}\FunctionTok{which.max}\NormalTok{(trig\_df}\SpecialCharTok{$}\NormalTok{y),], }
             \FunctionTok{aes}\NormalTok{(x,y,}\AttributeTok{label=}\FunctionTok{paste}\NormalTok{(}\StringTok{"max = ("}\NormalTok{,max\_x,}\StringTok{","}\NormalTok{,}\FunctionTok{round}\NormalTok{(max\_y,}\DecValTok{3}\NormalTok{),}\StringTok{")"}\NormalTok{)),}\AttributeTok{nudge\_y =}\NormalTok{ .}\DecValTok{1}\NormalTok{)}
\end{Highlighting}
\end{Shaded}

\includegraphics{R_Assignment_1a_files/figure-latex/test2c-1.pdf}

\begin{center}\rule{0.5\linewidth}{0.5pt}\end{center}

\hypertarget{section-3-8-points-this-problem-requires-finding-the-point-of-intersection-of-two-functions.-using-the-function-y-cosx-2-sinx-2-find-where-the-curved-line-y--x23-intersects-it-within-the-range-of-values-used-in-part-2-i.e.-4001-equally-spaced-values-from--2-to-2.-plot-both-functions-on-the-same-display-and-show-the-point-of-intersection.-present-the-coordinates-of-this-point-as-text-in-the-display.}{%
\subparagraph{\texorpdfstring{Section 3: (8 points) This problem
requires finding the point of intersection of two functions. Using the
function \texttt{y\ =\ cos(x\ /\ 2)\ *\ sin(x\ /\ 2)}, find where the
curved line y = -(x/2)\^{}3 intersects it within the range of values
used in part (2) (i.e.~4001 equally-spaced values from -2 to 2). Plot
both functions on the same display, and show the point of intersection.
Present the coordinates of this point as text in the
display.}{Section 3: (8 points) This problem requires finding the point of intersection of two functions. Using the function y = cos(x / 2) * sin(x / 2), find where the curved line y = -(x/2)\^{}3 intersects it within the range of values used in part (2) (i.e.~4001 equally-spaced values from -2 to 2). Plot both functions on the same display, and show the point of intersection. Present the coordinates of this point as text in the display.}}\label{section-3-8-points-this-problem-requires-finding-the-point-of-intersection-of-two-functions.-using-the-function-y-cosx-2-sinx-2-find-where-the-curved-line-y--x23-intersects-it-within-the-range-of-values-used-in-part-2-i.e.-4001-equally-spaced-values-from--2-to-2.-plot-both-functions-on-the-same-display-and-show-the-point-of-intersection.-present-the-coordinates-of-this-point-as-text-in-the-display.}}

\begin{Shaded}
\begin{Highlighting}[]
\FunctionTok{library}\NormalTok{(tidyverse)}
\NormalTok{func\_one }\OtherTok{\textless{}{-}} \ControlFlowTok{function}\NormalTok{(x) \{}
\NormalTok{  y }\OtherTok{\textless{}{-}} \FunctionTok{cos}\NormalTok{(x }\SpecialCharTok{/} \DecValTok{2}\NormalTok{) }\SpecialCharTok{*} \FunctionTok{sin}\NormalTok{(x }\SpecialCharTok{/} \DecValTok{2}\NormalTok{)}
  \FunctionTok{return}\NormalTok{(y)}
\NormalTok{\}}

\NormalTok{func\_two }\OtherTok{\textless{}{-}} \ControlFlowTok{function}\NormalTok{(x)\{}
\NormalTok{  y }\OtherTok{\textless{}{-}} \SpecialCharTok{{-}}\NormalTok{(x}\SpecialCharTok{/}\DecValTok{2}\NormalTok{)}\SpecialCharTok{\^{}}\DecValTok{3}
  \FunctionTok{return}\NormalTok{(y)}
\NormalTok{\}}

\NormalTok{x\_3 }\OtherTok{\textless{}{-}} \FunctionTok{seq}\NormalTok{(}\SpecialCharTok{{-}}\DecValTok{2}\NormalTok{, }\DecValTok{2}\NormalTok{, }\AttributeTok{length=}\DecValTok{4001}\NormalTok{)}

\NormalTok{y\_one }\OtherTok{\textless{}{-}} \FunctionTok{unlist}\NormalTok{(}\FunctionTok{lapply}\NormalTok{(x\_3,}\AttributeTok{FUN=}\NormalTok{func\_one))}

\NormalTok{y\_two }\OtherTok{\textless{}{-}} \FunctionTok{unlist}\NormalTok{(}\FunctionTok{lapply}\NormalTok{(x\_3,}\AttributeTok{FUN=}\NormalTok{func\_two))}

\NormalTok{sec3\_df }\OtherTok{\textless{}{-}} \FunctionTok{data.frame}\NormalTok{(x\_3,y\_one,y\_two)}

\NormalTok{intersection }\OtherTok{\textless{}{-}}\NormalTok{ sec3\_df[sec3\_df}\SpecialCharTok{$}\NormalTok{y\_one }\SpecialCharTok{==}\NormalTok{ sec3\_df}\SpecialCharTok{$}\NormalTok{y\_two,]}
\CommentTok{\#paste("intersection = (",intersection$x\_3,",",intersection$y\_one,")")}

\FunctionTok{ggplot}\NormalTok{(}\AttributeTok{data=}\NormalTok{sec3\_df, }\FunctionTok{aes}\NormalTok{(x\_3)) }\SpecialCharTok{+} 
  \FunctionTok{geom\_line}\NormalTok{(}\FunctionTok{aes}\NormalTok{(}\AttributeTok{y =}\NormalTok{ y\_one), }\AttributeTok{color =} \StringTok{"red"}\NormalTok{) }\SpecialCharTok{+}
  \FunctionTok{geom\_line}\NormalTok{(}\FunctionTok{aes}\NormalTok{(}\AttributeTok{y =}\NormalTok{ y\_two), }\AttributeTok{color =} \StringTok{"blue"}\NormalTok{) }\SpecialCharTok{+}
  \FunctionTok{geom\_point}\NormalTok{(}\FunctionTok{aes}\NormalTok{(}\AttributeTok{x=}\NormalTok{intersection}\SpecialCharTok{$}\NormalTok{x\_3, }\AttributeTok{y=}\NormalTok{intersection}\SpecialCharTok{$}\NormalTok{y\_one), }\AttributeTok{size =} \DecValTok{5}\NormalTok{, }\AttributeTok{colour=}\StringTok{"purple"}\NormalTok{) }\SpecialCharTok{+}
  \FunctionTok{geom\_text}\NormalTok{(}\AttributeTok{data=}\NormalTok{sec3\_df,}\FunctionTok{aes}\NormalTok{(}\AttributeTok{x=}\NormalTok{intersection}\SpecialCharTok{$}\NormalTok{x\_3, }\AttributeTok{y=}\NormalTok{intersection}\SpecialCharTok{$}\NormalTok{y\_one, }
            \AttributeTok{label=}\FunctionTok{paste}\NormalTok{(}\StringTok{"intersection = ("}\NormalTok{,intersection}\SpecialCharTok{$}\NormalTok{x\_3,}\StringTok{","}\NormalTok{,intersection}\SpecialCharTok{$}\NormalTok{y\_one,}\StringTok{")"}\NormalTok{),}
            \AttributeTok{vjust =} \SpecialCharTok{{-}}\DecValTok{3}\NormalTok{)}
\NormalTok{            )}
\end{Highlighting}
\end{Shaded}

\includegraphics{R_Assignment_1a_files/figure-latex/test3-1.pdf}

\begin{center}\rule{0.5\linewidth}{0.5pt}\end{center}

\hypertarget{section-4-12-points-use-the-trees-dataset-for-the-following-items.-this-dataset-has-three-variables-girth-height-volume-on-31-felled-black-cherry-trees.}{%
\subparagraph{Section 4: (12 points) Use the ``trees'' dataset for the
following items. This dataset has three variables (Girth, Height,
Volume) on 31 felled black cherry
trees.}\label{section-4-12-points-use-the-trees-dataset-for-the-following-items.-this-dataset-has-three-variables-girth-height-volume-on-31-felled-black-cherry-trees.}}

(4)(a) Use \emph{data(trees)} to load the dataset. Check and output the
structure with \emph{str()}. Use \emph{apply()} to return the median
values for the three variables. Output these values. Using R and
logicals, output the row number and the three measurements - Girth,
Height and Volume - of any trees with Girth equal to median Girth. It is
possible to accomplish this last request with one line of code.

\begin{Shaded}
\begin{Highlighting}[]
\FunctionTok{data}\NormalTok{(trees)}
\NormalTok{trees\_df }\OtherTok{=}\NormalTok{ trees}

\NormalTok{medians }\OtherTok{\textless{}{-}} \FunctionTok{apply}\NormalTok{(trees\_df,}\DecValTok{2}\NormalTok{,median)}

\CommentTok{\#get the index where girth == medians[1]}
\NormalTok{girth\_index }\OtherTok{\textless{}{-}} \FunctionTok{which}\NormalTok{(trees\_df}\SpecialCharTok{$}\NormalTok{Girth }\SpecialCharTok{==}\NormalTok{ medians[}\DecValTok{1}\NormalTok{])}
\CommentTok{\#print the rows from the df}
\NormalTok{girth\_rows }\OtherTok{\textless{}{-}}\NormalTok{ trees\_df[girth\_index,]}
\FunctionTok{print}\NormalTok{(girth\_rows)}
\end{Highlighting}
\end{Shaded}

\begin{verbatim}
##    Girth Height Volume
## 16  12.9     74   22.2
## 17  12.9     85   33.8
\end{verbatim}

\begin{Shaded}
\begin{Highlighting}[]
\CommentTok{\#do the above for height}
\NormalTok{height\_index }\OtherTok{\textless{}{-}} \FunctionTok{which}\NormalTok{(trees\_df}\SpecialCharTok{$}\NormalTok{Height }\SpecialCharTok{==}\NormalTok{ medians[}\DecValTok{2}\NormalTok{])}
\NormalTok{height\_rows }\OtherTok{\textless{}{-}}\NormalTok{ trees\_df[height\_index,]}
\FunctionTok{print}\NormalTok{(height\_rows)}
\end{Highlighting}
\end{Shaded}

\begin{verbatim}
##    Girth Height Volume
## 12  11.4     76   21.0
## 13  11.4     76   21.4
\end{verbatim}

\begin{Shaded}
\begin{Highlighting}[]
\CommentTok{\#do the above for volume}
\NormalTok{volume\_index }\OtherTok{\textless{}{-}} \FunctionTok{which}\NormalTok{(trees\_df}\SpecialCharTok{$}\NormalTok{Volume }\SpecialCharTok{==}\NormalTok{ medians[}\DecValTok{3}\NormalTok{])}
\NormalTok{volume\_rows }\OtherTok{\textless{}{-}}\NormalTok{ trees\_df[volume\_index,]}
\FunctionTok{print}\NormalTok{(volume\_rows)}
\end{Highlighting}
\end{Shaded}

\begin{verbatim}
##    Girth Height Volume
## 11  11.3     79   24.2
\end{verbatim}

\begin{Shaded}
\begin{Highlighting}[]
\CommentTok{\#one line output?}
\NormalTok{all\_rows }\OtherTok{\textless{}{-}}\NormalTok{ trees\_df[}\FunctionTok{c}\NormalTok{(girth\_index,height\_index,volume\_index),]}
\FunctionTok{print}\NormalTok{(all\_rows)}
\end{Highlighting}
\end{Shaded}

\begin{verbatim}
##    Girth Height Volume
## 16  12.9     74   22.2
## 17  12.9     85   33.8
## 12  11.4     76   21.0
## 13  11.4     76   21.4
## 11  11.3     79   24.2
\end{verbatim}

(4)(b) Girth is defined as the diameter of a tree taken at 4 feet 6
inches from the ground. Convert each diameter to a radius, r. Calculate
the cross-sectional area of each tree using pi times the squared radius.
Present a stem-and-leaf plot of the radii, and a histogram of the radii
in color. Plot Area (y-axis) versus Radius (x-axis) in color showing the
individual data points. Label appropriately.

\begin{Shaded}
\begin{Highlighting}[]
\FunctionTok{library}\NormalTok{(ggplot2)}
\NormalTok{trees\_df}\SpecialCharTok{$}\NormalTok{radius }\OtherTok{\textless{}{-}} \FunctionTok{with}\NormalTok{(trees\_df, Girth }\SpecialCharTok{/} \DecValTok{2}\NormalTok{)}

\NormalTok{trees\_df}\SpecialCharTok{$}\NormalTok{area }\OtherTok{\textless{}{-}} \FunctionTok{with}\NormalTok{(trees\_df, pi }\SpecialCharTok{*}\NormalTok{ radius }\SpecialCharTok{\^{}} \DecValTok{2}\NormalTok{)}

\CommentTok{\#make stem \& leaf}
\NormalTok{radius\_sl }\OtherTok{\textless{}{-}} \FunctionTok{stem}\NormalTok{(trees\_df}\SpecialCharTok{$}\NormalTok{radius)}
\end{Highlighting}
\end{Shaded}

\begin{verbatim}
## 
##   The decimal point is at the |
## 
##    4 | 234
##    5 | 34455667779
##    6 | 055799
##    7 | 013
##    8 | 0278
##    9 | 000
##   10 | 3
\end{verbatim}

\begin{Shaded}
\begin{Highlighting}[]
\NormalTok{radius\_sl}
\end{Highlighting}
\end{Shaded}

\begin{verbatim}
## NULL
\end{verbatim}

\begin{Shaded}
\begin{Highlighting}[]
\CommentTok{\#neat bin width calculation formula found online}
\NormalTok{radius\_bw }\OtherTok{\textless{}{-}} \DecValTok{2} \SpecialCharTok{*} \FunctionTok{IQR}\NormalTok{(trees\_df}\SpecialCharTok{$}\NormalTok{radius) }\SpecialCharTok{/} \FunctionTok{length}\NormalTok{(trees\_df}\SpecialCharTok{$}\NormalTok{radius)}\SpecialCharTok{\^{}}\NormalTok{(}\DecValTok{1}\SpecialCharTok{/}\DecValTok{3}\NormalTok{)}

\CommentTok{\#make the histogram}
\NormalTok{radius\_hist }\OtherTok{\textless{}{-}} \FunctionTok{ggplot}\NormalTok{(}\AttributeTok{data=}\NormalTok{trees\_df,}\FunctionTok{aes}\NormalTok{(}\AttributeTok{x=}\NormalTok{radius)) }\SpecialCharTok{+} 
  \FunctionTok{geom\_histogram}\NormalTok{(}\AttributeTok{fill=}\StringTok{"purple"}\NormalTok{, }\AttributeTok{binwidth =}\NormalTok{ radius\_bw) }\SpecialCharTok{+}
  \FunctionTok{stat\_bin}\NormalTok{(}\FunctionTok{aes}\NormalTok{(}\AttributeTok{y=}\NormalTok{..count.., }\AttributeTok{label=}\NormalTok{..count..), }\AttributeTok{binwidth =}\NormalTok{ radius\_bw, }\AttributeTok{geom=}\StringTok{"text"}\NormalTok{, }\AttributeTok{vjust=}\SpecialCharTok{{-}}\NormalTok{.}\DecValTok{5}\NormalTok{) }

\NormalTok{radius\_hist}
\end{Highlighting}
\end{Shaded}

\begin{verbatim}
## Warning: The dot-dot notation (`..count..`) was deprecated in ggplot2 3.4.0.
## i Please use `after_stat(count)` instead.
\end{verbatim}

\includegraphics{R_Assignment_1a_files/figure-latex/test3b-1.pdf}

\begin{Shaded}
\begin{Highlighting}[]
\CommentTok{\#make the scatter plot}
\NormalTok{radius\_area\_chart }\OtherTok{\textless{}{-}} \FunctionTok{ggplot}\NormalTok{(}\AttributeTok{data=}\NormalTok{trees\_df,}\FunctionTok{aes}\NormalTok{(}\AttributeTok{x=}\NormalTok{radius,}\AttributeTok{y=}\NormalTok{area)) }\SpecialCharTok{+}
  \FunctionTok{geom\_point}\NormalTok{(}\AttributeTok{size =} \DecValTok{4}\NormalTok{, }\AttributeTok{color =} \StringTok{"pink"}\NormalTok{)}
\NormalTok{radius\_area\_chart}
\end{Highlighting}
\end{Shaded}

\includegraphics{R_Assignment_1a_files/figure-latex/test3b-2.pdf}

(4)(c) Present a horizontal, notched, colored boxplot of the areas
calculated in (b). Title and label the axis.

\begin{Shaded}
\begin{Highlighting}[]
\NormalTok{area\_box }\OtherTok{\textless{}{-}} \FunctionTok{boxplot}\NormalTok{(trees\_df}\SpecialCharTok{$}\NormalTok{area,}
                    \AttributeTok{horizontal=}\ConstantTok{TRUE}\NormalTok{, }
                    \AttributeTok{notch=}\ConstantTok{TRUE}\NormalTok{,}
                    \AttributeTok{col=}\StringTok{"green"}\NormalTok{, }
                    \AttributeTok{xlab =} \StringTok{"area"}\NormalTok{)}
\FunctionTok{title}\NormalTok{(}\StringTok{"area boxplot"}\NormalTok{)}
\end{Highlighting}
\end{Shaded}

\includegraphics{R_Assignment_1a_files/figure-latex/test3c-1.pdf}

(4)(d) Demonstrate that the outlier revealed in the boxplot of Area is
not an extreme outlier. (Note: Extreme outlier is defined as a value
that falls outside the boundaries of Q1- 3\emph{IQR and Q3+ 3}IQR. Note
that R uses Q1 -1.5\emph{IQR and Q3+1.5}IQR as default values to
identify outliers). It is possible to do this with one line of code
using \emph{boxplot.stats()} or `manual' calculation and logicals.
Identify the tree with the largest area and output on one line its row
number and three measurements.

\begin{Shaded}
\begin{Highlighting}[]
\NormalTok{outlier }\OtherTok{\textless{}{-}}\NormalTok{ area\_box}\SpecialCharTok{$}\NormalTok{out}
\NormalTok{outlier\_row }\OtherTok{\textless{}{-}}\NormalTok{ trees\_df[}\FunctionTok{which}\NormalTok{(trees\_df}\SpecialCharTok{$}\NormalTok{area }\SpecialCharTok{==}\NormalTok{ outlier),]}
\NormalTok{outlier\_row}
\end{Highlighting}
\end{Shaded}

\begin{verbatim}
##    Girth Height Volume radius     area
## 31  20.6     87     77   10.3 333.2916
\end{verbatim}

\begin{Shaded}
\begin{Highlighting}[]
\CommentTok{\#one line using max()}
\NormalTok{max\_area\_row }\OtherTok{\textless{}{-}}\NormalTok{ trees\_df[}\FunctionTok{which.max}\NormalTok{(trees\_df}\SpecialCharTok{$}\NormalTok{area),]}
\NormalTok{max\_area\_row}
\end{Highlighting}
\end{Shaded}

\begin{verbatim}
##    Girth Height Volume radius     area
## 31  20.6     87     77   10.3 333.2916
\end{verbatim}

\begin{center}\rule{0.5\linewidth}{0.5pt}\end{center}

\hypertarget{section-5-12-points-the-exponential-distribution-is-an-example-of-a-right-skewed-distribution-with-outliers.-this-problem-involves-comparing-it-with-a-normal-distribution-which-typically-has-very-few-outliers.}{%
\subparagraph{Section 5: (12 points) The exponential distribution is an
example of a right-skewed distribution with outliers. This problem
involves comparing it with a normal distribution which typically has
very few
outliers.}\label{section-5-12-points-the-exponential-distribution-is-an-example-of-a-right-skewed-distribution-with-outliers.-this-problem-involves-comparing-it-with-a-normal-distribution-which-typically-has-very-few-outliers.}}

5(a) Use \emph{set.seed(124)} and \emph{rexp()} with n = 100, rate = 5.5
to generate a random sample designated as y. Generate a second random
sample designated as x with \emph{set.seed(127)} and \emph{rnorm()}
using n = 100, mean = 0 and sd = 0.15.

Generate a new object using \emph{cbind(x, y)}. Do not output this
object; instead, assign it to a new name. Pass this object to
\emph{apply()} and compute the inter-quartile range (IQR) for each
column: x and y. Use the function \emph{IQR()} for this purpose. Round
the results to four decimal places and present (this exercise shows the
similarity of the IQR values.).

For information about \emph{rexp()}, use \emph{help(rexp)} or
\emph{?rexp()}. \textbf{Do not output x or y.}

\begin{Shaded}
\begin{Highlighting}[]
\FunctionTok{set.seed}\NormalTok{(}\DecValTok{124}\NormalTok{)}
\NormalTok{y\_5 }\OtherTok{\textless{}{-}} \FunctionTok{rexp}\NormalTok{(}\DecValTok{100}\NormalTok{,}\AttributeTok{rate =} \FloatTok{5.5}\NormalTok{)}

\FunctionTok{set.seed}\NormalTok{(}\DecValTok{127}\NormalTok{)}
\NormalTok{x\_5 }\OtherTok{\textless{}{-}} \FunctionTok{rnorm}\NormalTok{(}\DecValTok{100}\NormalTok{,}\AttributeTok{mean =} \DecValTok{0}\NormalTok{, }\AttributeTok{sd =} \FloatTok{0.15}\NormalTok{)}

\NormalTok{xy\_5 }\OtherTok{\textless{}{-}} \FunctionTok{cbind}\NormalTok{(x\_5,y\_5)}

\CommentTok{\#type 6 gets the 3 digit round to be equal, and need round to 4 digit}
\CommentTok{\#default type 7 doesn\textquotesingle{}t get us close per the instruction}
\NormalTok{iqr\_5 }\OtherTok{\textless{}{-}} \FunctionTok{apply}\NormalTok{(xy\_5,}\DecValTok{2}\NormalTok{,IQR, }\AttributeTok{type =} \DecValTok{6}\NormalTok{)}

\FunctionTok{round}\NormalTok{(iqr\_5,}\AttributeTok{digits =} \DecValTok{4}\NormalTok{)}
\end{Highlighting}
\end{Shaded}

\begin{verbatim}
##    x_5    y_5 
## 0.2151 0.2206
\end{verbatim}

(5)(b) This item will illustrate the difference between a right-skewed
distribution and a symmetric one. For base R plots, use \emph{par(mfrow
= c(2, 2))} to generate a display with four diagrams;
\emph{grid.arrange()} for ggplots. On the first row, for the normal
results, present a histogram and a horizontal boxplot for x in color.
For the exponential results, present a histogram and a horizontal
boxplot for y in color.

\begin{Shaded}
\begin{Highlighting}[]
\FunctionTok{par}\NormalTok{(}\AttributeTok{mfrow =} \FunctionTok{c}\NormalTok{(}\DecValTok{2}\NormalTok{, }\DecValTok{2}\NormalTok{))}
\FunctionTok{hist}\NormalTok{(x\_5, }\AttributeTok{xlab =} \StringTok{"normal distribution"}\NormalTok{, }\AttributeTok{col =} \StringTok{"purple"}\NormalTok{, }\AttributeTok{main =} \StringTok{"Normal Histogram"}\NormalTok{)}
\FunctionTok{boxplot}\NormalTok{(x\_5, }\AttributeTok{horizontal =} \ConstantTok{TRUE}\NormalTok{, }\AttributeTok{xlab =} \StringTok{"normal distribution"}\NormalTok{, }\AttributeTok{col =} \StringTok{"purple"}\NormalTok{, }\AttributeTok{main =} \StringTok{"Normal Box Plot"}\NormalTok{)}
\FunctionTok{hist}\NormalTok{(y\_5, }\AttributeTok{xlab =} \StringTok{"exponential distribution"}\NormalTok{, }\AttributeTok{col =} \StringTok{"gold"}\NormalTok{, }\AttributeTok{main =} \StringTok{"Exponential Histogram"}\NormalTok{)}
\FunctionTok{boxplot}\NormalTok{(y\_5, }\AttributeTok{horizontal =} \ConstantTok{TRUE}\NormalTok{, }\AttributeTok{xlab =} \StringTok{"exponential distribution"}\NormalTok{, }\AttributeTok{col =} \StringTok{"gold"}\NormalTok{, }\AttributeTok{main =} \StringTok{"Exponetial Box Plot"}\NormalTok{)}
\end{Highlighting}
\end{Shaded}

\includegraphics{R_Assignment_1a_files/figure-latex/test5b-1.pdf}

(5)(c) QQ plots are useful for detecting the presence of heavy-tailed
distributions. Present side-by-side QQ plots, one for each sample, using
\emph{qqnorm()} and \emph{qqline()}. Add color and titles. In base R
plots, ``cex'' can be used to control the size of the plotted data
points and text; ``size'' for ggplot2 figures. Lastly, determine if
there are any extreme outliers in either sample.Remember extreme
outliers are based on 3 multiplied by the IQR in the box plot. R uses a
default value of 1.5 times the IQR to define outliers (not extreme) in
both boxplot and boxplot stats.

\begin{Shaded}
\begin{Highlighting}[]
\FunctionTok{library}\NormalTok{(glue)}
\FunctionTok{par}\NormalTok{(}\AttributeTok{mfrow =} \FunctionTok{c}\NormalTok{(}\DecValTok{1}\NormalTok{, }\DecValTok{2}\NormalTok{))}
\CommentTok{\#thin tail due to left end being above and right end being below}
\FunctionTok{qqnorm}\NormalTok{(x\_5,}\AttributeTok{col =} \StringTok{"blue"}\NormalTok{, }\AttributeTok{main =} \StringTok{"Norm Dist QQ Plot"}\NormalTok{)}
\FunctionTok{qqline}\NormalTok{(x\_5, }\AttributeTok{col =} \StringTok{"red"}\NormalTok{)}
\CommentTok{\#right skewed since curve above line both ends}
\FunctionTok{qqnorm}\NormalTok{(y\_5, }\AttributeTok{col =} \StringTok{"green"}\NormalTok{, }\AttributeTok{main =} \StringTok{"Expo Dist QQ Plot"}\NormalTok{)}
\FunctionTok{qqline}\NormalTok{(y\_5, }\AttributeTok{col =} \StringTok{"red"}\NormalTok{)}
\end{Highlighting}
\end{Shaded}

\includegraphics{R_Assignment_1a_files/figure-latex/test5c-1.pdf}

\begin{Shaded}
\begin{Highlighting}[]
\CommentTok{\#get q1 and q3 for x then calc bounds}
\NormalTok{x\_5\_quantiles }\OtherTok{\textless{}{-}} \FunctionTok{c}\NormalTok{(}\FunctionTok{quantile}\NormalTok{(x\_5,}\FloatTok{0.25}\NormalTok{),}\FunctionTok{quantile}\NormalTok{(x\_5,}\FloatTok{0.75}\NormalTok{))}
\NormalTok{x\_5\_3iqr }\OtherTok{\textless{}{-}}\NormalTok{ (x\_5\_quantiles[}\DecValTok{2}\NormalTok{] }\SpecialCharTok{{-}}\NormalTok{ x\_5\_quantiles[}\DecValTok{1}\NormalTok{]) }\SpecialCharTok{*} \DecValTok{3}
\NormalTok{x\_5\_bounds }\OtherTok{\textless{}{-}} \FunctionTok{c}\NormalTok{((x\_5\_quantiles[}\DecValTok{1}\NormalTok{] }\SpecialCharTok{{-}}\NormalTok{  x\_5\_3iqr),(x\_5\_quantiles[}\DecValTok{2}\NormalTok{] }\SpecialCharTok{+}\NormalTok{ x\_5\_3iqr))}
\CommentTok{\#get q1 and q3 for y x then calc bounds}
\NormalTok{y\_5\_quantiles }\OtherTok{\textless{}{-}} \FunctionTok{c}\NormalTok{(}\FunctionTok{quantile}\NormalTok{(y\_5,}\FloatTok{0.25}\NormalTok{),}\FunctionTok{quantile}\NormalTok{(y\_5,}\FloatTok{0.75}\NormalTok{))}
\NormalTok{y\_5\_3iqr }\OtherTok{\textless{}{-}}\NormalTok{ (y\_5\_quantiles[}\DecValTok{2}\NormalTok{] }\SpecialCharTok{{-}}\NormalTok{ y\_5\_quantiles[}\DecValTok{1}\NormalTok{]) }\SpecialCharTok{*} \DecValTok{3}
\NormalTok{y\_5\_bounds }\OtherTok{\textless{}{-}} \FunctionTok{c}\NormalTok{((y\_5\_quantiles[}\DecValTok{1}\NormalTok{] }\SpecialCharTok{{-}}\NormalTok{  y\_5\_3iqr),(y\_5\_quantiles[}\DecValTok{2}\NormalTok{] }\SpecialCharTok{+}\NormalTok{ y\_5\_3iqr))}

\CommentTok{\#x outliers are empty}
\NormalTok{x\_5\_ext\_outliers }\OtherTok{\textless{}{-}}\NormalTok{ xy\_5[}\FunctionTok{which}\NormalTok{(x\_5 }\SpecialCharTok{\textless{}}\NormalTok{ x\_5\_bounds[}\DecValTok{1}\NormalTok{] }\SpecialCharTok{|}\NormalTok{ x\_5 }\SpecialCharTok{\textgreater{}}\NormalTok{ x\_5\_bounds[}\DecValTok{2}\NormalTok{])]}
\FunctionTok{glue}\NormalTok{(}\StringTok{\textquotesingle{}extreme outliers for x: \{x\_5\_ext\_outliers\}\textquotesingle{}}\NormalTok{)}

\NormalTok{y\_5\_ext\_outliers }\OtherTok{\textless{}{-}}\NormalTok{ xy\_5[}\FunctionTok{which}\NormalTok{(y\_5 }\SpecialCharTok{\textless{}}\NormalTok{ y\_5\_bounds[}\DecValTok{1}\NormalTok{] }\SpecialCharTok{|}\NormalTok{ y\_5 }\SpecialCharTok{\textgreater{}}\NormalTok{ y\_5\_bounds[}\DecValTok{2}\NormalTok{])]}
\FunctionTok{glue}\NormalTok{(}\StringTok{\textquotesingle{}extreme outliers for y: \{y\_5\_ext\_outliers\}\textquotesingle{}}\NormalTok{)}
\end{Highlighting}
\end{Shaded}

\begin{verbatim}
## extreme outliers for y: -0.0469025955882937
\end{verbatim}

\end{document}
